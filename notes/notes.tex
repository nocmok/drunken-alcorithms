\documentclass[a4paper,12pt]{article}

%%% Работа с русским языком
\usepackage{cmap}					% поиск в PDF
\usepackage{mathtext} 				% русские буквы в формулах
\usepackage[T2A]{fontenc}			% кодировка
\usepackage[utf8]{inputenc}			% кодировка исходного текста
\usepackage[english,russian]{babel}	% локализация и переносы

\usepackage{etoolbox} % логические операторы
\usepackage{layout}

\usepackage{fancyhdr}

%%% Математика
\usepackage{amsmath,amsfonts,amssymb,amsthm,mathtools}
\mathtoolsset{showonlyrefs=true}

\begin{document}
	\
	\vfil
	\hfil Конспекты \hfil
	\vfil
	
	\newpage
	\tableofcontents
	
	\newpage
	\section{Матеша}
	\subsection{Логарифмы}
	Полезные свойства логарифмов:
	\begin{equation}
	\boxed{
		log_a(n^k) = k \cdot log_a(n)
	}
	\end{equation}
	Доказательство:
	\begin{eqnarray}
		a^{log_a(n^k)} = n^k = (a^{log_a(n)})^k = a^{k\cdot log_a(n)} \\ \blacksquare
	\end{eqnarray}
	\begin{equation}
	\boxed{
		log_a(n) = \frac{log_b(n)}{log_b(a)}
	}
	\end{equation}
	Доказательство:
	\begin{eqnarray}
		n = a^{log_a(n)} = b^{log_b(n)} \Rightarrow log_b(n) = log_b(a^{log_a(n)}) = log_a(n)\cdot log_b(a) \Rightarrow \\ \Rightarrow log_a(n) = \frac{log_b(n)}{log_b(a)} \\ \blacksquare
	\end{eqnarray}
	
	\begin{equation}
		\boxed{
			log_{a^b}(n) = \frac{1}{b} \cdot log_a(n)
		}
	\end{equation}
	Доказательство:
	\begin{eqnarray}
		log_{a^b}(n) = \frac{log_{a}(n)}{log_a(a^b)} = \frac{1}{b} \cdot log_a(n) \\ \blacksquare
	\end{eqnarray}
	\begin{equation}
		\boxed{
			a^{log_b(n)} = n^{log_b(a)}	
		}
	\end{equation}
	Доказательство:
	\begin{eqnarray}
		a^{log_b(n)} = n^{log_b(a)}	\Leftrightarrow log_a(a^{log_b(n)}) = log_a(n^{log_b(a)}) \Leftrightarrow \\ \Leftrightarrow log_b(n) \cdot log_a(a) = log_b(a) \cdot log_a(n) \Leftrightarrow log_a(n) = \frac{log_b(n)}{log_b(a)} \\ \blacksquare
	\end{eqnarray}
	
	
	\subsection{Основная теорема о рекурентных соотношениях (разделяй и властвуй)}
	Предположим, некоторый алгоритм который для решения задачи размером $n$ сначала рекурсивно вызывает сам себя $a$ раз на подзадачах размером $\frac{n}{b}$, а также тратит время $O(n^d)$ на подготовку к рекурсии и сборку ответа. Тогда время работы алгоритма:
	\begin{eqnarray}
		T(n) = a\cdot T(\lceil \frac{n}{b} \rceil) + O(n^d), a > 0, b > 1, d \geq 0
	\end{eqnarray}
	\begin{eqnarray}
		T(n) & = & O(n^d),~ d > log_b(a) \\
		T(n) & = & O(n^d \cdot log(n)),~ d = log_b(a) \\
		T(n) & = & O(n^{log_b(a)}), ~ d < log_b(a)
	\end{eqnarray}
\end{document}